% !TEX program = xelatex
\documentclass{article}
\usepackage[UTF8]{ctex}
\usepackage[margin=1in]{geometry}
\usepackage{float}
\usepackage{amsfonts,amssymb} 
\usepackage{bm}
\usepackage{amsmath}
\usepackage{color}

\title{Neo Hookean UMAT推导}
\author{徐辉}
\begin{document}
\maketitle
\section{符号说明}
下面列出推导中常用的变量转化关系和符号说明:

变形梯度
$$\bm{F}=\frac{\partial \bm{x}}{ \partial \bm{X}}$$

左Cauchy Green变形张量
$$\bm{B}= \bm{F} \cdot \bm{F}^{\rm T}$$

右Cauchy Green变形张量
$$\bm{C}=\bm{F}^{\rm T}\cdot \bm{F}$$

PK-2应力
$$\bm{S}$$

Cauchy应力
$$\bm{\sigma}=\frac{1}{J}\bm{F}\cdot \bm{S} \cdot \bm{F}^{\rm T}$$

Kirchhoff应力
$$\bm{\tau}=J\bm{\sigma}$$
材料的Jacobian满足
$$\delta(J\bm{\sigma})=J\mathbb{C}\cdot \delta \bm{D}$$

虚变形率
$$\delta \bm{D}={\rm sym}(\delta \bm{F}\cdot \bm{F}^{-1})$$

不变量求导
$$\frac{\partial I_3}{\partial \bm{A}}=\bm{A}^{\rm -T}\quad \frac{\partial I_3}{\partial A_{ij}}=I_3 A_{ji}^{-1}$$

\section{基本推导}
\begin{equation*}
    \begin{aligned}
        \frac{\partial B_{ij}}{\partial F_{kl}} & =\frac{\partial (F_{im}F_{mj}^T)}{\partial F_{kl}}=\frac{\partial (F_{im}F_{jm})}{\partial F_{kl}}
        =\frac{\partial F_{im}}{\partial F_{kl}}F_{jm} +F_{im} \frac{\partial F_{jm}}{\partial F_{kl}}                                               \\
                                                & =\delta_{ik}\delta_{ml}F_{jm}+\delta_{jk}\delta_{ml}F_{im}                                         \\
                                                & =\delta_{ik}F_{jl}+\delta_{jk}F_{il}
    \end{aligned}
\end{equation*}
$$\frac{\partial \ln J}{\partial F_{kl}}=\frac{1}{J}\frac{\partial J}{\partial F_{kl}}=\frac{1}{J}(JF_{lk}^{-1})=F_{lk}^{-1}$$
\section{材料Jacobian}

虚变形率表达式带入到材料的Jacobian中,写成分量形式有
$$\delta(J\sigma_{ij})=J \mathbb{C}_{ijkl}\ {\rm sym} (\delta F_{km}F_{ml}^{-1})$$
在计算$\mathbb{C}$的时候,需要对变形率进行求导。首先暂时忽略$\delta \bm{D}$的对称性,然后将计算结果对称化保证对称性。先暂时忽略对称性,有
$$\delta(J\sigma_{ij})=J \mathbb{C}_{ijkl} \delta F_{km}F_{ml}^{-1}$$
进行如下操作(??)
\begin{equation*}
    \begin{aligned}
        \delta(J\sigma_{ij})&=J \mathbb{C}_{ijkl} \delta F_{km}F_{ml}^{-1}             \\
        \delta(J\sigma_{ij})F_{lp}&=J \mathbb{C}_{ijkl} \delta F_{km}F_{ml}^{-1}F_{lp} \\
        \delta(J\sigma_{ij})F_{lp}&=J \mathbb{C}_{ijkl} \delta F_{km}\delta_{mp}             \\
        \delta(J\sigma_{ij})F_{lm}&=J \mathbb{C}_{ijkl} \delta F_{km}                  \\
    \end{aligned}
\end{equation*}
移项有
$$\mathbb{C}_{ijkl}=\frac{1}{J}\frac{\partial(J\sigma_{ij})}{\partial F_{km}}F_{lm}$$
强制保证$\mathbb{C}$的对称性,有
$$\mathbb{C}_{ijkl}=\frac{1}{2J}\left(\frac{\partial(J\sigma_{ij})}{\partial F_{km}}F_{lm} + \frac{\partial(J\sigma_{ij})}{\partial F_{lm}}F_{km} \right)$$

\section{Abaqus Neo-Hookean模型推导}

Abaqus超弹性neo-hookean模型应变能函数\cite{abaqus}
$$U=C_{10}(\bar{I}_1-3)+\frac{1}{D_1}(J-1)^2$$
Cauchy应力为
$$\sigma_{ij}=\frac{2}{J}C_{10}(\bar{B}_{ij}-\frac{1}{3}\delta_{ij}\bar{B}_{kk})+\frac{2}{D_1}(J-1)\delta_{ij}$$
因此
$$J\sigma_{ij}=2C_{10}(\bar{B}_{ij}-\frac{1}{3}\delta_{ij}\bar{B}_{kk})+\frac{2}{D_1}(J^2-J)\delta_{ij}$$

计算第一部分
$$\frac{\partial (J\sigma_{ij})}{\partial F_{km}}=\frac{\partial}{\partial F_{km}}\left(2C_{10}(\bar{B}_{ij}-\frac{1}{3}\delta_{ij}\bar{B}_{kk})\right)=\cdots$$

计算第二部分
$$\frac{\partial (J\sigma_{ij})}{\partial F_{km}}=\frac{\partial}{\partial F_{km}}\left(\frac{2}{D_1}(J^2-J)\delta_{ij}\right)=\cdots$$


\section{分子链 Neo-Hookean模型推导}
基于分子链模型应变能函数
$$U=\frac{1}{2}\mu_0({\rm trace}(\bm{C})-3)-\mu_0 \ln J+\frac{1}{2}\lambda_0(\ln J)^2$$
Kirchhoff应力为\cite{bely}(5.4.55)
$$\bm{\tau}=\mu_0(\bm{B}-\bm{I}) + \lambda_0 \ln J \bm{I}$$
写作分量形式为
$$\tau_{ij}=\mu_0(B_{ij}-\delta_{ij}) + \lambda_0 \ln J \delta_{ij}$$
Cauchy应力为
$$\bm{\sigma}=\frac{\tau}{J}=\frac{\mu_0}{J}(\bm{B}-\bm{I}) + \frac{\lambda_0}{J} \ln J \bm{I}$$
写作分量形式为
$$\sigma_{ij}=\frac{\mu_0}{J}(B_{ij}-\delta_{ij}) + \frac{\lambda_0}{J} \ln J \delta_{ij}$$
计算材料的Jacobian
$$\frac{\partial (J\sigma_{ij})}{\partial F_{km}}=\frac{\partial \tau_{ij}}{\partial F_{km}}=\mu_0(\delta_{ik}F_{jm}+\delta_{jk}F_{im}) + \lambda_0\delta_{ij}F_{mk}^{-1}$$
\begin{equation*}
    \begin{aligned}
        \frac{\partial (J\sigma_{ij})}{\partial F_{km}}F_{lm} & =\mu_0(\delta_{ik}F_{jm}F_{lm}+\delta_{jk}F_{im}F_{lm}) + \lambda_0\delta_{ij}F_{mk}^{-1}F_{lm} \\
                                                              & =\mu_0(\delta_{ik}B_{jl}+\delta_{jk}B_{il}) + \lambda_0\delta_{ij}F_{mk}^{-1}F_{lm}
    \end{aligned}
\end{equation*}
同理有
\begin{equation*}
    \begin{aligned}
        \frac{\partial (J\sigma_{ij})}{\partial F_{lm}}F_{km} & =\mu_0(\delta_{il}B_{jk}+\delta_{jl}B_{ik}) + \lambda_0\delta_{ij}F_{ml}^{-1}F_{km}
    \end{aligned}
\end{equation*}
可以得到材料的Jacobian
$$\mathbb{C}_{ijkl}=\frac{1}{2J}\left(\mu_0(\delta_{ik}B_{jl}+\delta_{jk}B_{il}+\delta_{il}B_{jk}+\delta_{jl}B_{ik}) +\lambda_0\delta_{ij}(F_{mk}^{-1}F_{lm}+F_{ml}^{-1}F_{km}) \right)$$



\begin{thebibliography}{1}
    \bibitem{huang} 高等固体力学 @ 黄克智
    \bibitem{kzb} 非线性连续介质力学@匡震邦
    \bibitem{abaqus} abaqus
    \bibitem{bely}  Nonlinear Finite Elements for Continua and Structures @ Belytschko
\end{thebibliography}

\end{document}